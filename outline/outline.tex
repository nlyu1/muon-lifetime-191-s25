\documentclass[aps,prb,onecolumn,11pt,superscriptaddress,floatfix,longbibliography]{revtex4-2}

\usepackage{amsmath,amssymb} % math symbols
\usepackage{bm} % bold math font
\usepackage{graphicx} % for figures
\usepackage{comment} % allows block comments

\usepackage{enumitem}
\setlist{noitemsep,leftmargin=*,topsep=0pt,parsep=0pt}

\usepackage{hyperref}

\usepackage{xcolor}
\definecolor{blue}{RGB}{33, 118, 199}
\newcommand{\blue}[1]{\textcolor{blue}{#1}}
\definecolor{green}{RGB}{0, 128, 0}
\newcommand{\green}[1]{\textcolor{green}{#1}}
\definecolor{red}{RGB}{230, 0, 20}
\newcommand{\red}[1]{\textcolor{red}{#1}}

\begin{document}

\title{Muon Lifetime Experiment Summary}

\author{Nicholas Lyu}
\email[]{nicholaslyu@college.harvard.edu}
\affiliation{Harvard University}
\author{Adam Pearl}
\email[]{apearl@college.harvard.edu}
\affiliation{Harvard University}

\date{January 25, 2022}

\maketitle

\section{\label{sec:goals}Experiment Goals}

The primary objective of this experiment is to measure the lifetime of a muon (\(\tau_\mu\)) by observing its decay and fitting the resulting exponential distribution. This measurement also allows us to infer the Fermi constant (\(G_F\)) since \(\tau_\mu^{-1} \propto G_F^2\). Additionally, we aim to determine the muon rest mass by analyzing the energy spectrum of decay electrons.

\begin{enumerate}
    \item Measure the lifetime of a muon \(\tau_\mu\) and extract the weak force coupling constant \(G_F\).
    \begin{enumerate}
        \item Muons originate from cosmic ray interactions, where pions and kaons decay into muons and muon neutrinos.
        \item Muons decay via weak interaction: 
        \begin{align*}
            \mu^- &\to e^- \bar{\nu}_e \nu_\mu, \\
            \mu^+ &\to e^+ \nu_e \bar{\nu}_\mu.
        \end{align*}
        \item Negative muons can also be absorbed in matter: 
        \begin{align*}
            \mu^- p^+ \to n \nu_\mu.
        \end{align*}
        The probability of capture depends on the atomic number \(Z\) and scales roughly as \(Z^4\) for low \(Z\). Since we use a plastic (hydrocarbon) scintillator, we approximate it as carbon.
        \item The decay rate follows:  
        \[
        -\frac{dN}{dt} = N_0 \Gamma_\mu e^{-\Gamma_\mu t}.
        \]
        Taking the logarithm, we obtain a linear form:  
        \[
        \ln\left(-\frac{dN}{dt}\right) = \ln(N_0 \Gamma_\mu) - \Gamma_\mu t.
        \]
        We fit this to extract \(\Gamma_\mu\), from which \(\tau_\mu\) can be determined.
    \end{enumerate}
    
    \item Measure the muon rest mass by analyzing the decay electron energy spectrum.
    \begin{enumerate}
        \item In the decay \(\mu^- \to e^- \bar{\nu}_e \nu_\mu\), energy conservation dictates that the electron carries roughly half of the muon's rest energy.
        \item The energy spectrum of electrons is analyzed to identify the high-energy cutoff, which provides an estimate of the muon mass.
        \item Calibration is performed using the energy deposition (\(dE/dx\)) of minimally ionizing muons.
    \end{enumerate}
\end{enumerate}

\section{\label{sec:experiment}Experimental Considerations}

\begin{enumerate}
    \item \textbf{Trigger Calibration:}  
    The detection system must effectively filter out non-muon events. The muon signature is identified using:  
    \begin{itemize}
        \item \textbf{Start Signal:} \(T \wedge M \wedge \bar{B}\), indicating a muon stopping in the middle detector.
        \item \textbf{Stop Signal:} Any secondary signal in \(M\), or an event where the electron escapes upward (\(T \wedge M \wedge \bar{B}\)) or downward (\(\bar{T} \wedge M \wedge B\)).
    \end{itemize}
    A gate lasting several muon lifetimes is implemented, and the \(B\) signal is extended to prevent false triggers.

    \item \textbf{Scintillation and Detection:}  
    Muons excite the plastic scintillator, producing light that is detected by the photomultiplier tubes (PMTs).  
    \begin{itemize}
        \item The PMT converts light into an electrical signal via the photoelectric effect.
        \item The signal is amplified through a cascade of dynodes, creating a measurable output.
        \item Possible issues include afterpulsing from ionized residual gases inside the PMT.
    \end{itemize}

    \item \textbf{Data Acquisition and Analysis:}  
    The oscilloscope and LabVIEW software record time intervals between the start and stop signals. A histogram of these delays should follow an exponential decay, allowing \(\tau_\mu\) extraction through curve fitting.
\end{enumerate}

\section{\label{sec:safety}Safety Considerations}

\begin{enumerate}
    \item \textbf{High Voltages:}  
    \begin{itemize}
        \item PMTs operate at -2.0 kV to -2.4 kV. Handle cables carefully.
        \item DO NOT hot-plug high voltage connections.
    \end{itemize}
    \item \textbf{Radiation:}  
    \begin{itemize}
        \item Cosmic ray muons pose no significant health risk.
        \item Ensure proper shielding of electrical components.
    \end{itemize}
\end{enumerate}

\section{\label{sec:questions}Questions}

\begin{enumerate}
    \item Why do we only consider muons that stop in the middle detector for lifetime measurements?
    \item Are we fitting \(\tau_\mu\) from the decay rate equation, or are we directly measuring lifetimes from individual events?
    \item What are the primary sources of background noise in the signal, and how do we mitigate them?
    \item How does the discriminator threshold affect the efficiency of the detector, and how should it be calibrated?
    \item How does the afterpulsing effect in PMTs influence our measurements, and what techniques can be used to suppress it?
\end{enumerate}

% \nocite{BernheimAJP1956}  
% \nocite{Buhr1983}  
% \nocite{McMahon1983}  
% \nocite{MartinAJP1984}  
% \nocite{Liu1987}  
% \nocite{Hanne1988}  

% \bibliography{refs}

\end{document}
